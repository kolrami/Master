%!TEX root = thesis.tex
\begin{abstract}
Modern industrial applications include a broad range of requirements, far
exceeding the traditional control and automation tasks. To accomodate these
new challenges, the so-called \ac{PAC} was introduced, combining the
advantages of traditional \acp{PLC} and \acp{PC}. The recent versions,
additionally, incorporate custom hardware for high-performance or low-latency
tasks, implemented on a \ac{FPGA}. This thesis investigates the feasibility of
a heterogeneous reconfigurable hardware/software architecture (ReconOS) for
\acp{PAC}, with a focus on communication, programmability and self-adaptation
strategies.

The integration of \ac{HLS} approaches into the ReconOS toolchain reduces the
implementation effort for hardware threads considerably and lowers the entry
barrier for software developers. Combined with a newly developed hardware
interconnect for low-latency and jitter-free communication between hardware
threads providing a speedup of up to 100, the feasibility of the ReconOS
framework for modern \ac{PAC} devices was proven by a Ball on Plate
demonstrator. By incorporating hardware accelerated threads, a speedup of
around 14 could be achieved and the power consumption could be lowered by 4\%.
Furthermore, self-adaption strategies were investigated and integrated into
the demonstrator, adapting parameters of the system to optimize the energy
consumption.
\end{abstract}