%!TEX root = thesis.tex
\chapter{Conclusion}
\label{sec:conclusion}
\aclp{PAC} reflect the industries demand for more advanced and powerful
controllers and introduce new challenges for the future of control. To
compensate for the growing demand of features, these device incorporate
powerful embedded processing systems, \acp{FPGA} and \ac{COTS} components.
However, by introducing heavily heterogeneous systems with increasing
complexity, programming and maintaining becomes challenge for the developers.

This thesis evaluated new programming methods and adaption
techniques developed in the context of heterogeneous embedded systems,
investigated their feasibility and introduced new concepts for their use in
\acp{PAC}. Therefore, the thesis included three main subtasks, motivated from
the posed problem definition. In a first step, the programmability of these
devices is addressed, by applying the multithreaded programming model based on
the ReconOS framework and extending it with support for \ac{HLS}. By
integrating it seamlessly into the existing toolchain and providing a unified
program description for both hardware and software implementation , its
feasibility and advantages for several applications was proven.

Based on the extended ReconOS framework, shortcomings of the current
implementation were addressed, limiting the benefits for its usage in
\ac{PAC} applications demanding low-latency and low-jitter executions. To
compensate the overhead introduced by the delegate mechanism without loosing
its complete transparency against the implementation modes of threads, a
reconfigurable interconnect in combination with hardware resources was
developed. It allows hardware threads to communicate directly, without
interfacing the processor and reduces the communication overhead and jitter.
Based on an evaluation of related work, the new interconnect architecture was
proposed and extensively evaluated. By comparing several aspects of an
application, the benefits of the new interconnect were clearly stated.
Additionally, with regard to current research on self-adapting systems,
considerations for reconfiguring the interconnect were made and a working
prototype was presented.

Since the self-adapting capabilities are highly relevant in current research
and also a for future \ac{PAC} applications, the third part of this theses
addressed this topic. Therefore, and to demonstrate the use of all proposed
techniques, a Ball on Plate demonstrator was developed, including combinations
of hard- and software threads implemented traditionally and with the use of
\ac{HLS}, the new interconnect architecture, as well as self-adaption
strategies to minimize power consumption. An extensive evaluation of this
system was presented to prove the feasibility of the presented approaches in a
real-world \ac{PAC} application.

\section{Future Work}
While this thesis introduced new ideas and developed prototypical
demonstrations, different aspects might be used as a basis for future research
and engineering work. While the \ac{HLS} introduction has proven its
feasibility, it might provide even further features. Since the states of a
thread are generated out of a high level description, \ac{HLS} approaches
might allow to automatically handle state migration by generating logic to
save the thread's state. This would allow to schedule the hardware threads in
case of, for example, insufficient device resources without additional effort
by the developer.

Also the proposed interconnect architecture provides possibilities for further
research. A detailed study of different types of interconnects, for example
buses or rings, and their suitability for different kinds of applications is
missing. Furthermore, the reconfiguration of the interconnect could be
evaluated more detailed with different types of interconnects. Additionally,
the reconfiguration of hardware resources was totally out of the scope of this
thesis, but is directly related to the reconfiguration of the interconnect.
Just the same as for hardware threads, the resources' state must be saved and
restored after migration.

As pointed out in the evaluation of the demonstrator, possible self-adaption
strategies for this application are only limited, at least to influence the
power consumption. A more detailed analysis to what extends real-world
applications can benefit from self-adaptive capabilities and how they can be
utilized in a hybrid multi-core system is still part of ongoing research.