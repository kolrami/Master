%!TEX root = thesis.tex
\chapter{Introduction}
%\epigraphhead[60]{\epigraph{``The future for PACs hinges on the incorporation
%of embedded technology''}{\textit{\citet{bel05}}}}

Modern industrial applications include a broad range of requirements, far
exceeding the traditional control and automation tasks. Buzzwords like
\emph{Industry 4.0} or the \emph{Internet of Things} emphasize the growing
demand of features. While a typical control system interfaces with simple
sensors and actuators, modern applications require more sophisticated features
like network connectivity, integration into enterprise systems or device
interoperability. To accommodate these new challenges and since the nowadays
dominating \ac{PLC} reaches its limits, the \ac{PAC} was introduced as the
next generation of control. It combines the reliability and robustness of a
\ac{PLC} with the advanced software capabilities of a \ac{PC}. By
incorporating state-of-the-art embedded technology like \acp{FPGA} and its
integration with general purpose processors on a single chip, these devices
eliminate the need for an expensive custom hardware and allow  the use of
\ac{CTOS} components, without loosing the benefits of executing time critical
control functions directly in silicon.

Unfortunately, the increasing complexity of such systems introduces new
challenges and demand convenient programming methods abstracting the
underlying low-level components, to make these devices accessible to a broad
range of developers. Classical ladder logic used for \acp{PLC} reaches its
limits and \ac{FPGA} technology is restricted to hardware designers capable of
handling low level programming languages like VHDL. Because of its acceptance
in the field of software engineering and its eligibility to model embedded
systems composed of concurrently running tasks, the multithreaded approach
seems to be suitable as a convenient programming model for these hybrid
devices. The different threads specified by the developer are distributed
across the system's \ac{CPU} and \ac{FPGA} resources and communicate via
defined interaction mechanisms like message passing or shared memory.

\todo{Proposal Andreas}
\begin{itemize}
\item Background for PAC and ReconOS
\item Improvements on ReconOS for PAC
\item Ease of programming by HLS
\item Demonstrator to show feasability
\end{itemize}
