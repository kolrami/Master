%!TEX root = thesis.tex
\chapter{Introduction}
%\epigraphhead[60]{\epigraph{``The future for PACs hinges on the incorporation
%of embedded technology''}{\textit{\citet{bel05}}}}

Modern industrial applications include a broad range of requirements, far
exceeding the traditional control and automation tasks. Emerging trends like
\emph{Industry 4.0} or the \emph{Internet of Things} emphasize the growing
demand of features. While a typical control system interfaces with simple
sensors and actuators, modern applications require more sophisticated features
like network connectivity, integration into enterprise systems or device
interoperability. To accommodate these new challenges and, since the nowadays
dominating \ac{PLC} reaches its limits, the \ac{PAC} was introduced as the
next generation of control. It combines the reliability and robustness of a
\ac{PLC} with the advanced software capabilities of a \ac{PC}. Incorporating
state-of-the-art embedded technology like \acp{FPGA} eliminates the need for
expensive custom hardware designs, without loosing the benefits of executing
time critical control functions directly in silicon.

Unfortunately, the increasing complexity of such systems introduces new
challenges and demands convenient programming methods, abstracting the
underlying low-level components. Classical ladder logic used for \acp{PLC}
reaches its limits and \ac{FPGA} technology is restricted to hardware
designers capable of handling low level programming languages like \ac{VHDL}.
The lack of programmability limits the acceptance among the existing
developers and prevents a wider dissemination. Because of its acceptance in
the field of software engineering and its eligibility to model embedded
systems composed out of concurrently running tasks, the multithreaded approach
seems to be suitable as a convenient programming model for these hybrid
devices. The different threads specified by the developer are distributed
across the system's \ac{CPU} and \ac{FPGA} resources and communicate via
defined interaction mechanisms like message passing or shared memory. Combined
with recent advances in \ac{HLS} techniques and low latency communication
infrastructures, this approach seems to be suitable for modern \ac{PAC}
designs.

Besides the challenges in programming these hybrid systems, their increasing
complexity and dynamics pose problems for the maintainers and administrators.
Rapidly changing environmental conditions and varying demands require the
systems to self-adapt without the interference of a human, incapable of
surveying the entire system and all its coherences.

This thesis investigates the feasibility of the multithreaded approach as
implemented by the ReconOS framework for \ac{PAC} designs and studies
different self-adaptation strategies for a prototypical application. Chapter
\ref{chap:motivation} motivates the problem definition and introduces relevant
concepts and fundamentals. Chapter \ref{chap:hls} introduces the Vivado HLS
tool and presents an integration into the ReconOS framework to generate
hardware threads including communication and processing logic, followed by
chapter \ref{chap:interconnect}, addressing the problem of communication
overhead of threads by proposing a hardware interconnect infrastructure. Based
on these results, chapter \ref{chap:demonstrator} presents a real-world Ball
on Plate application, demonstrating the use of ReconOS and applying different
self-adaptation strategies.