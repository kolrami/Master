%!TEX root = thesis.tex
\chapter{Demonstrator}
\label{sec:demo}

To demonstrate the work described in the previous chapters in a real world
application, a ball on plate system is presented in the following sections.
The demonstrator tries to balance a ball by tracking its position and
adjusting the angle of the underlying plate, thus allowing to compensate
external impacts and keeping the ball in the center. Therefore, the plate
itself is mounted on top of a Stewart platform and has a resistive touchscreen
attached to it. While the Stewart platform offers the ability to rotate the
plate around arbitrary axes, the touchscreen provides precise and low latency
positioning information of the ball.

The control of the platform is implemented by a \ac{PAC} based on the previous
considerations, including analog I/O, low latency data paths and control
loops, as well as self-adaptive capabilities. The different components of the
system are described in detail in the following sections. These include the
kinematics of the Stewart platform, the touchscreen interface and control
algorithms for balancing the ball.

\section{Stewart Platform}
The Stewart platform was introduced by D. Stewart in 1965 \citep{Ste65} and
gained popular resarch interest in robotics \citep{Szu13}, concerning
kinematics, dynamics or work space estimations. However, outside the
scientific community the Stewart platform was not widely adopted and most
designs are constrained to applications requiring \ac{6DoF}, for instance
flight simulators or \ac{CNC} machining centers. Although a Stewart platform
provides many potential advantages , sophisticated synthesis and simulation
tools are missing \citep{Ji96}. Against the background of increasing
computational capabilities and more efficient \ac{CAD} tools, this situation
might change in the future. In the scope of this thesis, the Stewart platform
promises to be a sophisticated development platform, involving computationally
expensive calculations for its kinematics and need for additional I/O
capabilities.

The following sections introduce the architecture of the Stewart platform in
general and the variant used in this thesis, specifying the used notation and
solving the inverse kinematics problem.

\subsection{Architecture}
While Stewart introduced the general idea of a Stewart platform in his work in
1965 \citep{Ste65}, a strict definition is missing in the literature. The only
commonality is the characterization as a parallel manipulator \citep{Szu13}
consisting out of a bottom base and an upper platform. Compared to a serial
manipulator, it consists out of multiple serial chains controlled
simultaneously and connected to a single end-effector. While this design
provides unique benefits like high accuracy, better load-to-weight ratios and
improved rigidity, it limits the available work space and causes highly
nonlinear behavior.

%\begin{figure}
%	\centering
%	\begin{subfigure}{0.3\textwidth}
%		\centering
%		\includegraphics[width=\textwidth]{../figures/stewart_architectures_c}
%		\caption{6-UPS}
%		\label{fig:stewart_architectures_a}
%	\end{subfigure}
%	\begin{subfigure}{0.3\textwidth}
%		\centering
%		\includegraphics[width=\textwidth]{../figures/stewart_architectures_d}
%		\caption{Fixed prismatic}
%		\label{fig:stewart_architectures_b}
%	\end{subfigure}
%	\begin{subfigure}{0.3\textwidth}
%		\centering
%		\includegraphics[width=\textwidth]{../figures/stewart_architectures_e}
%		\caption{Fixed rotary}
%		\label{fig:stewart_architectures_c}
%	\end{subfigure}
%	\caption{Different variants of Stewart platforms \citep[adopted from][]{Szu13}}
%	\label{fig:stewart_architectures}
%\end{figure}
The different variant of a Stewart platform vary in the type of the
connections and actuators, as well as their spacial configuration, i.e. the
positioning of the different connections. The most common realization utilizes
six prismatic actuator, for example hydraulic pistons or linear motors. It is
often referred to as Hexapod and associated with the term Stewart platform,
not least because of its similarity to the original design proposed by
Stewart. Other architectures utilize fixed prismatic or rotary actuators as to
vary the length of the six legs. All of these variants have its unique
benefits and drawbacks. Since the actuators in the latter design can be
realizing as commonly available servo motors, it gained popularity, especially
for low-cost designs and first prototypes. Therefore, this thesis focuses on a
servo-based architecture as shown in figure
\ref{fig:stewart}. The mechanical design and construction was not done in this
thesis, but adopted from a previous work.
\begin{figure}
	\centering
	\includegraphics[width=6cm]{../figures/stewart}
	\caption{Construction of the Stewart platform prototype}
	\label{fig:stewart}
\end{figure}

\subsection{Construction and Notation}
This section introduces the basic notations, coordinate systems and dimensions
of the Steward platform used in this thesis. Consider figure
\ref{fig:stewart_notation}, showing different views annotated with relevant
notations.
\begin{figure}
	\centering
	\begin{subfigure}{0.49\textwidth}
		\centering
		\includegraphics{../figures/stewart_base}
		\caption{Top view of base}
		\label{fig:stewart_base}
	\end{subfigure}
	\begin{subfigure}{0.49\textwidth}
		\centering
		\includegraphics{../figures/stewart_platform}
		\caption{Top view of platform}
		\label{fig:stewart_platform}
	\end{subfigure}
	\par\bigskip
	\begin{subfigure}{0.49\textwidth}
		\centering
		\includegraphics{../figures/stewart_side}
		\caption{Side view}
		\label{fig:stewart_side}
	\end{subfigure}
	\caption{Notations for the Stewart platform}
	\label{fig:stewart_notation}
\end{figure}

As already described previously, the system consists out of a lower base and
an upper platform, connected by six legs. For both, the base and platform, a
particular coordinate system is introduced as shown in figures
\ref{fig:stewart_base} and \ref{fig:stewart_platform}. While origin of the
base coordinate system $B$ is located in the center of the base and at the
same height as the shaft of the servos, the origin of platform's coordinate
system $P$ is located exactly above at the height of the platform joints
$J_i$. With regard to the platform's coordinate system, the coordinates of the
six joints are annotated in figure \ref{fig:stewart_platform}.

Additionally, each servo specifies its own coordinate system $S_i$, located at
the intersection point of the servo's shaft and the servo's joint position.
Thereby, the y axis points away from the base and the x axis to servo's joint
in neutral position. Resulting from these two axes, the z axis points up or
down, depending of the servo, i.e. for even servos it points down and for odd
servos it point up. Figure \ref{fig:stewart_side} illustrates the servo
coordinate systems again and showing the angle $\alpha$ of the servo arm. The
different coordinate systems for even and odd servos allow to handle all
servos analogously and respects the physical configuration of the servo arms.

Calculating the inverse kinematics of the platform requires to transform the
joints of the platform to the corresponding servo coordinate system. This is
done in two steps, by transforming them to $B$, and finally to $S_i$. While
the transformation to the base coordinate system $B$ is trivial, by simply
adding an offset $C_h$ to the z coordinate, transforming the resulting point
to the servo coordinate system $S_i$ requires more sophisticated calculations,
including rotation and scaling. At first, the base coordinate system is
rotated to match the y axis of the appropriate servo coordinate system,
followed by mirroring the x and z axis to fit the desired orientations.
Finally, the origin is translated to match the servo positions by adding and
subtracting $C_{sx}$ and $C_{sy}$ to the x and y coordinates, respectively.

\subsection{Inverse Kinematics}
Taking the basic notations of the previous section as a basis, the inverse
kinematic calculates the required servo positions for a desired angle of the
platform. Unlike for serial manipulators, the inverse kinematics of a Stewart
platform has a unique analytic solution, while the forward kinematics problem
is highly nonlinear and requires iterative approaches. However, the low-cost
variant utilizing rotary instead of linear actuators introduces additional
calculations to calculate the required angles from the desired leg lengths.

Assuming, that the desired angle of the platform is given by a rotation axis
$r$ and a rotation angle $r_a$, equation \ref{eq:platform_rot} shows the
calculation of the required platform joint's positions.
\begin{equation}
J'_i = 
\begin{pmatrix}
\left[\left(1 - \cos(r_a)\right) r_x r_x + \cos(r_a)\right] J_ix + \left[\left(1 - \cos(r_a)\right) r_x r_y\right] J_iy\\
\left[\left(1 - \cos(r_a)\right) r_x r_y\right] J_ix + \left[\left(1 - \cos(r_a)\right) r_y r_y + \cos(r_a)\right] J_iy\\
\left[-r_y \sin(r_a)\right] J_ix + \left[r_x \sin(r_a)\right] J_iy\\
\end{pmatrix}
\label{eq:platform_rot}
\end{equation}
Transforming these position to the appropriate servo coordinate systems as
described in the previous section allows to calculate the distance between the
servo joint and the platform joint. While for a linear actuator this would be
the solution, rotary actuators can not vary the length of the legs directly.
Therefore, the angle $\alpha$ must be adjusted in a way, that the distance
between its joint and the platform joint is equal to the fixed leg length
$C_l$. While there exist an analytic solution for this problem, it includes
terms of higher degree and complex operations like $\arctan$ and square roots.
Therefore, an iterative approach was chosen to find the appropriate angle.
More details on the concrete implementation of this iterative approach will be
covered in a following sections.

\section{Touchscreen}
To track the ball's position on the plate, a resistive touchscreen is mounted
on top of the plate and touched by the ball. Compared to, for example camera
based tracking techniques, a touchscreen provides low latency data independent
from external influences like lighting conditions or camera movements.
Furthermore, it requires less computational effort, as well as a more compact
setup.

The following subsections introduce briefly the fundamental principles of the
resistive touchscreen technology and covers the use in the prototype.

\subsection{Resistive Touchscreen Technology}
An analog resistive touchscreen consists out a glass substrate and a flexible
film, both coated with a conductor \ac{ITO} \citep{Wal12}, as shown in figure
\ref{fig:touch_build}.
\begin{figure}
	\centering
	\begin{subfigure}{0.49\textwidth}
		\centering
		\includegraphics[width=\textwidth]{../figures/touch_build}
		\caption{Construction of resistive touchscreen}
		\label{fig:touch_build}
	\end{subfigure}
	\begin{subfigure}{0.49\textwidth}
		\centering
		\includegraphics[width=\textwidth]{../figures/touch_five}
		\caption{Five-wire touchscreen}
		\label{fig:touch_five}
	\end{subfigure}
	\caption{Analog resistive touchscreen technology}
	\label{fig:touch}
\end{figure}
The two conductive surfaces are facing each other and
are separated by insulating spacer dots \citep{Wal12}. When a voltage is
applied across the surfaces and the flexible film is pressed down, the two
surfaces make electrical contact, resulting in a voltage divider created by
the resistance of the \ac{ITO}. Measuring the ratio of the different voltages
allows to calculate the touch position.

Basically, there exist three major variants of resistive touchscreens, namely
four-wire, five-wire and eight-wire touchscreens, differing in the number of
connections to the sensor \citep{Wal12}. For the demonstrator, a five-wire
touchscreen was used as shown in figure \ref{fig:touch_five}, due to its
increased durability and sustainability. The X and Y voltages are applied to
the four corners of the lower glass, while the upper film is used only as a
contact point to measure the voltage at the touch position (wiper). To
determine, for example, the X position, a voltage is applied to the two left
corners, while two right corners are connected to ground.

\subsection{Interfacing the Touchscreen}
To interface the touchscreen as stated in the previous section, voltage needs
to be applied to different corners of the touchscreen, while all other ones
needs to be connected to ground. Therefore, regular digital I/O are
sufficient, setting their output to 1 or 0, respectively. To measure the
voltage at the touch point through the viper, an \ac{ADC} with suitable input
ranges is required. To eliminate noise caused by an unstable voltage source,
the \ac{ADC} should be connected in a differential fashion to the wiper and
the supply voltage. This improves the reliability and minimizes the effects of
voltage fluctuations \citep{OOD00}.

In this work, the pre-assembled evaluation module ADS7845EVM \citep{ADS06} was
used. While the ZedBoard provides both digital I/O and an on-chip \ac{ADC}
which could be utilized to interface a touchscreen, additional circuits would
be required to meet the \ac{ADC}'s input requirements. Since circuit design is
not target of this thesis, a pre-assembled evaluation module was chosen. It
provides an appropriate connector for the touchscreen and interfaces with the
\ac{FPGA} via the \ac{SPI} interface as shown in figure \ref{fig:touch_spi}.
\begin{figure}
	\centering
	\includegraphics{../figures/touch_spi}
	\caption{\acs{SPI} interface of the ADS7845 touch controller}
	\label{fig:touch_spi}
\end{figure}
A single conversion is done in 24 clock cycles with a period of
$\SI{400}{\nano\second}$ by sending eight configuration bits to the controller
and receiving the converted voltage at the touch position as a 12 bit integer.
The configuration bits send in the first byte include the channel selection
\emph{A2} - \emph{A0}, i.e. the x and y directions, mode selections and power
down options. The \emph{SD} bit selects between single-ended and differential
mode, \emph{M} selects between eight and twelve bit conversion and \emph{P1} -
\emph{P0} control the power down between conversions. After the control byte
is send, the controller enters the conversion mode and performs the analog-to-
digital conversion in the next twelve clock cycles. In the 13th clock cycle
the last bit is transferred, followed by three zero bits to complete the last
byte. Therefore, 48 clock cycles are necessary to read the x and y coordinate
of the touch position, i.e. around $\SI{20}{\micro\second}$.

To connect the controller evaluation module to the ZedBoard, regular digital
I/Os are required for the serial interface, as well as a $\SI{3.3}{\volt}$
voltage supply, provided by one of the \ac{Pmod} connectors.

\section{Control Algorithm}

\section{Implementation}